%\documentclass[onecolumn, 12pt]{article}
%\documentclass[onecolumn, 12pt]{book}
%
%\usepackage[latin1]{inputenc}   
%\usepackage{amsmath}
%\usepackage{algorithm}
%\usepackage{algorithmic} 
%\usepackage[T1]{fontenc}
%
%\usepackage[francais]{babel}     
%\usepackage{layout}    
%\usepackage[top=2cm, bottom=2cm, left=2cm, right=2cm]{geometry} 
%\usepackage{setspace}
%\usepackage{soul}
%\usepackage{color} 
%\usepackage{verbatim}
%\usepackage{moreverb}
%\usepackage{listings}
%\usepackage{url}
%\usepackage{graphicx}
%\usepackage{epstopdf}
%\usepackage{caption}
%\usepackage{setspace}
% 
% 
% \title{Mod\`ele de Donn\'ees}
% \author{Willy Ehounou}
% %\date{01/06/15}
%\title{Plan de th\`ese}
%\author{Jules \bsc{Verne}}
%\date{\oldstylenums{1875}} 


 
\begin{document}
\maketitle
\tableofcontents

\chapter{Introduction G\'en\'erale}
\chapter{Contexte de l'\'etude}
	\section{Etat de l'art: reconstruction du graphe par les incidents}
	Dans cette partie, nous utilisons le cas du r\'eseau \'electrique de ERDF, d\'ecouvert \`a partir des incidents. Nous d\'ecrivons les \'etapes pour parvenir \`a ce resultat. 
	
	\section{Probl\'ematique}
	Nous definirons le probl\`eme proxi-line qui est la distance de Hamming minimale selon les conditions suivantes:
	
\chapter{Reseau de flots et Mesures }
	
	\section{Graphe de flots: R\'eseau \'electrique}
		\subsection{Mod\'elisation du r\'eseau \'electrique}
	\section{Grandeurs physiques et caracteristiques}
	\section{Loi de conservation dans un r\'eseau \'electrique }
	\section{Contraintes dans les r\'eseaux \'electriques}
\chapter{Corr\'elation de mesures}
	\section{Mesures: des s\'eries temporelles}
		\subsection{S\'eries Temporelles: D\'efinitions et Propri\'et\'es}
		\subsection{Particularit\'es des mesures}
%		\subsection{}
	\section{\'Etude de l'art sur la corr\'elation de s\'eries temporelles}
		\subsection{Corr\'elation sur les s\'eries enti\`eres: \{W,D\}DTW, TWE, MSM, CID, DTDc, COTE}
			\subsubsection{Dynamic Time Warping: weight WDTW, derivative DDTW }
			\subsubsection{Time Warp Edit TWE}
			\subsubsection{Move-split-merge MSM}
			\subsubsection{Complexity invariant Distance CID}
			\subsubsection{Derivative Transform Distance DTDc}
			\subsubsection{Elastic Ensemble EE}
			\subsubsection{Collective of Transform Ensembles COTE}
			\subsubsection{}
		\subsection{Corr\'elation par intervalles: TSF, TSBF, LPS}
			\subsubsection{Time Series Forest TSF}
			\subsubsection{Time Series Bag of Features TSBF}
			\subsubsection{Learned Pattern Similarity LPS}
		\subsection{Corr\'elation par parties significatives(shapelets): FS, ST, LS}
			\subsubsection{Fast Shapelets FS}
			\subsubsection{Shapelet Transform ST}
			\subsubsection{Learned Shapelets LS}
		\subsection{Corr\'elation par agr\'egation des features: BOP, SAXVSM, BOSS,  DTW$_{F}$}
			\subsubsection{Bag of Pattern BOP}
			\subsubsection{Symbolic Aggregate approXimation Vectr Space Model SAXVSM}
			\subsubsection{Bag of Symbolic Fourier Approximation (SFA) symbols BOSS}
			\subsubsection{Dynamic Time Warping Features DTW$_{F}$}
	\section{Proposition d'une m\'ethode de corr\'elation de mesures \'electriques}
	\section{Formalisation et calcul de la Matrice de corr\'elation}
\chapter{Algorithmes de graphes}
	\section{Matrice de Corr\'elation: un Line graph}
		\subsection{Line graphs: D\'efinitions et Propri\'et\'es}
		\subsection{Particularit\'es de la matrice de corr\'elation}
	\section{\'Etude de l'art sur la d\'ecomposition d'un Line graph}
	\section{Proposition d'algorithmes de resolution de line graph}
		\subsection{Algorithme de couverture du Line graph}
		\subsection{Algorithme de correction du Line graph}
		\subsection{Application sur un graphe Iourte}
		\subsection{Complexit\'e des algorithmes}
	\section{Construction du r\'eseau \'electrique et Orientation des ar\^etes }
\chapter{simulation avec des donn\'ees r\'eelles}
	\section{R\'eseau \'electrique: datacenter d'un op\'erateur de t\'el\'ecommunications }
	\section{Grandeurs et Mesures}
	\section{Matrice de corr\'elation}
	\section{Application des algorithmes de couverture et de correction}
	\section{Discussion sur le r\'eseau propos\'e}
\chapter{Conclusions}
\chapter{Perpectives}
\tableofcontents
 \end{document}