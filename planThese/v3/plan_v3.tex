%\documentclass[onecolumn, 12pt]{article}
\documentclass[onecolumn, 12pt]{book}

\usepackage[latin1]{inputenc}   
\usepackage{amsmath}
\usepackage{algorithm}
\usepackage{algorithmic} 
%\usepackage[T1]{fontenc}

%\usepackage[francais]{babel}     
\usepackage{layout}    
\usepackage[top=2cm, bottom=2cm, left=2cm, right=2cm]{geometry} 
\usepackage{setspace}
\usepackage{soul}
\usepackage{color} 
\usepackage{verbatim}
\usepackage{moreverb}
\usepackage{listings}
\usepackage{url}
\usepackage{graphicx}
\usepackage{epstopdf}
\usepackage{caption}
\usepackage{setspace}
 
 
% \title{Mod\`ele de Donn\'ees}
% \author{Willy Ehounou}
 %\date{01/06/15}
\title{Plan de th\`ese}
%\author{Jules \bsc{Verne}}
\date{\oldstylenums{\today}} 

%---insert paragraph (use 4) and subparagraph (use 5) to table of contents
\setcounter{tocdepth}{4} 
\setcounter{secnumdepth}{4}

 
\begin{document}
\maketitle
\tableofcontents

\chapter{Introduction G\'en\'erale}
\chapter{Contexte de l'\'etude}
	\section{\'Etat de l'art: reconstruction du graphe par les incidents}
	Dans cette partie, nous utilisons le cas du r\'eseau \'electrique de ERDF, d\'ecouvert \`a partir des incidents. Nous d\'ecrivons les \'etapes pour parvenir \`a ce resultat. 
	
	\section{Probl\'ematique}
	Nous d\'efinirons le probl\`eme proxi-line qui est la distance de Hamming minimale selon certaines conditions.
	
\chapter{R\'eseau de flots et Mesures }
	\section{R\'eseau \'electrique d'un datacenter: un graphe de flots}
	Dans cette section, nous allons decrire un reseau de datacenter et faire le lien entre ce reseau et un graphe de flots.
	\section{Mod\'elisation du r\'eseau \'electrique}
	Nous allons decrire les grandeurs presentes dans le datacenter et leurs mesures associ\'ees. 
	Les mesures des grandeurs physiques constituent le flot dans le graphe. 
	Nous parlerons des contraintes li\'ees aux flots dans le reseau \'electriques. Ces contraintes ont le nom de loi de conservation ou de kirrchhoff( loi des noeuds(I,P) et des mailles (U)).
		\subsection{Grandeurs physiques}
		\subsection{Mesures physiques : des Series Temporelles}
			nous definirons une serie temporelle, donnerons ses proprietes et ses avantages.
		\subsection{Constraintes dans un reseau \'electrique : Loi de Kirchhoff}
	
\chapter{Corr\'elation de mesures}
	\section{Analyse de s\'eries temporelles}
	Ici nous parlerons des types d'analyses qu'on realise avec les series temporelles.
	Nous indiquerons le type d'analyses que nous souhaiterons faire sur nos mesures et le domaine qui s'y rattache. 
		\subsection{S\'eries Temporelles : domaines d'analyses}
		\subsection{Pourquoi analyser nos mesures comme des series temporelles}
	\section{\'Etude de l'art sur la corr\'elation de s\'eries temporelles}
		\subsection{Corr\'elation sur les s\'eries enti\`eres: \{W,D\}DTW, TWE, MSM, CID, DTDc, COTE}
			\subsubsection{Dynamic Time Warping: weight WDTW, derivative DDTW }
			\subsubsection{Time Warp Edit TWE}
			\subsubsection{Move-split-merge MSM}
			\subsubsection{Complexity invariant Distance CID}
			\subsubsection{Elastic Ensemble EE}
			\subsubsection{Collective of Transform Ensembles COTE}
			\subsubsection{}
		\subsection{Corr\'elation par intervalles: TSF, TSBF, LPS}
			\subsubsection{Time Series Forest TSF}
			\subsubsection{Time Series Bag of Features TSBF}
			\subsubsection{Learned Pattern Similarity LPS}
		\subsection{Corr\'elation par parties significatives(shapelets): FS, ST, LS}
			\subsubsection{Fast Shapelets FS}
			\subsubsection{Shapelet Transform ST}
			\subsubsection{Learned Shapelets LS}
		\subsection{Corr\'elation par agr\'egation des features: BOP, SAXVSM, BOSS,  DTW$_{F}$}
			\subsubsection{Bag of Pattern BOP}
			\subsubsection{Symbolic Aggregate approXimation Vectr Space Model SAXVSM}
			\subsubsection{Bag of Symbolic Fourier Approximation (SFA) symbols BOSS}
			\subsubsection{Dynamic Time Warping Features DTW$_{F}$}
	\section{Proposition d'une m\'ethode de corr\'elation de mesures \'electriques}
	\section{Formalisation et calcul de la Matrice de corr\'elation}
	
\chapter{Matrice de correlation : Un line graphe }
	\section{\'Etat de l'art: les line graphes}
	\section{Proposition d'algorithmes}
		\subsection{Line-couverture}
		\subsection{Algorithme de couverture}
		\subsection{Algorithme de correction}
		\subsection{Complexit\'e des algorithmes}
	\section{D\'etermination de la topologie du r\'eseau \'energetique}
	\section{Correction particuli\`ere de Graphes : Graphes Iourtes }

\chapter{Evaluation des algorithmes sur des donn\'ees theoriques et r\'eelles}
	\section{Simulation des algorithmes sur des r\'eseaux th\'eoriques}
		\subsection{Objectifs et d\'efinitions}
		\subsection{Donn\'ees: G\'en\'eration al\'eatoires de graphes}
		\subsection{G\'en\'eration de line graphes sous jacent aux r\'eseaus de flots non orient\'e}
		\subsection{Prise en compte de l'erreur de corr\'elation dans la matrice $matE$}
		\subsection{R\'esultats}
			\subsubsection{Distribution de la m\'ethode de permutation al\'eatoire}
			\subsubsection{Relation entre la distance-line et la distance de Hamming}
			\subsubsection{Comparaison des m\'ethodes de correction}
			\subsubsection{Influence des erreurs de corr\'elations sur les distributions}
			\subsubsection{Impact de la fonction de co\^ut sur les distributions}
	\section{Simulation sur le datacenter Champlan}
		\subsection{Description du datacenter Champlan}
		\subsection{Matrice de corr\'elation : corr\'elation fausses n\'egatives et fausses positives}
		\subsection{Line-couverture et correction de la matrice de corr\'elation}
		\subsection{Discussion sur le r\'eseau propos\'e}
	
\chapter{Conclusions}
\chapter{Perpectives}
\tableofcontents
 \end{document}