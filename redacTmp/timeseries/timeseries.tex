\documentclass[onecolumn, 12pt]{book}

\usepackage[latin1]{inputenc}   
\usepackage{amsmath}
\usepackage{algorithm}
\usepackage{algorithmic} 
%\usepackage[T1]{fontenc}

%\usepackage[francais]{babel}     
\usepackage{layout}    
\usepackage[top=2cm, bottom=2cm, left=2cm, right=2cm]{geometry} 
\usepackage{setspace}
\usepackage{soul}
\usepackage{color} 
\usepackage{verbatim}
\usepackage{moreverb}
\usepackage{listings}
\usepackage{url}
\usepackage{graphicx}
\usepackage{epstopdf}
\usepackage{caption}
\usepackage{setspace}
 
 
% \title{Mod\`ele de Donn\'ees}
% \author{Willy Ehounou}
 %\date{01/06/15}
\title{redaction time series}
\author{Wilfried Ehounou}
\date{\oldstylenums{\today}} 

\newtheorem{definition}{D\'efinition}
\newtheorem{property}{Propri\'et\'e}
\newtheorem{theorem}{Theorem}
\newtheorem{claim}[theorem]{Claim}
\newtheorem{proposition}[theorem]{Proposition}
\newtheorem{lemma}[theorem]{Lemma}
\newtheorem{corollary}[theorem]{Corollary}
\newtheorem{conjecture}[theorem]{Conjecture}
\newtheorem{observation}{Observation}
\newtheorem{example}{Exemple}
\newtheorem{remark}{Remark}

%---- path figures ----
\graphicspath{{/home/willy/Documents/courbePython/courbeDegreCoutMinAleatoire_11_09_2017/}
{/home/willy/Documents/courbePython/courbeDegreCoutMinAleatoire_11_10_2017/}
{/home/willy/Documents/courbePython/courbeDegreCoutMinAleatoire_11_09_2017/comparaison_MethodesCorrection_fctDeCout_permut_aleatoire_coutMin_degreMin/}{/home/willy/Documents/latexDoc/redactionThese/chap3_linegraphs/imagesLineGraphes/}}
 
\begin{document}
\maketitle
\tableofcontents

\chapter{Mesures : Des Series Temporelles}

\section{Series temporelles}

\begin{definition}
Une s\'erie temporelle est une suite chronologique de valeurs r\'eelles $x_t$ \`a des instants de temps \'ecart\'es r\'eguli\`erement. 
% formule
\begin{equation}
(x_t)_{1 \le t \le n}
\end{equation}
avec $n$ la dimension de la s\'erie.
\end{definition}
L'intervalle de temps entre deux mesures successives d\'epend de la s\'erie. 
Il peut s'agir d'un jour, d'une semaine, d'une minute ...
\newline
G\'en\'eralement, les s\'eries temporelles s'utilisent dans les probl\`emes suivants :
\begin{itemize}
	\item mod\'elisation : la representation de la serie sous la forme d'une fonction du temps.
	\item pr\'evision : pr\'edire les donn\'ees futures \`a partir de valeurs pr\'ed\'ecentes.
	\item changement de r\'egime :  la s\'erie change-t-elle significativement \`a un instant $t$.
	\item comparaison :  determiner la relation existante entre une s\'erie observ\'ee et d'autres s\'eries candidates.
\end{itemize}
Dans ce document, nous nous concentrerons \`a la derni\`ere categorie des probl\`emes. 

Selon le but de l'etude \`a realiser, deux definitions s'imposent.

\subsection{Particularit\'es des series temporelles}
Bien que la s\'erie temporelle $(x_t)_{1 \le t \le n}$ est l'observation des $n$ premi\`eres r\'ealisations d'un processus stochastique $(X_t)_t$, elle se modelise sous la forme suivante:
\begin{definition}
la s\'erie temporelle se d\'ecompose en une somme de trois composantes 
\begin{itemize}
	\item deterministe ou tendance  non nulle $m$
	\item p\'eriodique ou saisonni\`ere $s$
	\item al\'eatoire ou le bruit $\epsilon$
\end{itemize}
\begin{equation}
 X_t = m + s + \epsilon = f(t) + \epsilon
\end{equation}
\end{definition}
\begin{definition}
La fonction $f(t)$ est decompos\'ee en deux termes: 
\begin{equation}
	f(t) = m(t) + S(t)
\end{equation}
\`ou $m(t)$ et $S(t)$ sont deux fonctions avec 
\begin{enumerate}
	\item  $S(t)$ est une fonction p\'eriodique non nulle de p\'eriode $T$ telle que $\sum_{i = 1}^{T}S(i) = 0$ alors $s$ est la composante saisonni\`ere de $X$.
	\item si $m(t)$ est non nulle alors $m$ est la tendance de $X$.
\end{enumerate}
avec $S(i)$ les coefficients de la fonction $S$.
\end{definition}
La tendance est une fonction d'une famille simple d\'ecrite par peu de param\`etres : fonction affine (droite), polyn\^ome de degr\'e faible, ou fonction exponentielle. 
Le choix des familles de fonctions parmi lesquelles on cherche la fonction $f$
se fait souvent \`a vue, en regardant les graphiques des donn\'ees sans autre justifcation
th\'eorique : certaines saisonalit\'es sont visibles, d'autres apparaissent plus nettement apr\`es certaines transformations des donn\'ees brutes (transform\'ee de Fourier appel\'ee p\'eriodogramme). 
Une tendance croissante r\'eguli\`ere est mod\'elis\'ee par une tendance affine, par un polyn\^ome d'ordre $2$ si on observe un certain creusement... 
Il n'existe pas de crit\`ere objectif pour faire le choix de cette famille. 
\newline
Le choix de la p\'eriode de la saisonalit\'e est orient\'e par des connaissances a priori sur la s\'erie. 
Par exemple, les ph\'enom\`enes li\'es aux climats, ainsi qu'une grande partie des s\'eries \'economiques ont une p\'eriodicit\'e annuelle.
Mais une s\'erie de consommation \'electrique pr\'esente une saisonnalit\'e hebdomadaire en plus.
\newline
Certaines s\'eries se mod\'elisent par un produit des trois composantes. Cependant, l'application d'un logarithme nous permet de revenir \`a la propri\'et\'e pr\'ec\'edente.
\begin{equation}
 Y_t = log(X_t) = log(m) + log(s) + log(\epsilon)
\end{equation}
Le mod\`ele multiplicatif est approch\'e au mod\`ele additif.

% definitions de m_t, s_t et \epsilon et methodes de calcul de ces composantes
\subsection{Estimation de la tendance et la saisonnalit\'e}
Les methodes present\'ees s'utilisent generalement dans la pr\'evision de donn\'ees.
Nous utilisons ces m\'ethodes dans le but de d\'eterminer des param\^etres qui permettront de s'approcher, au pire des cas, de la s\'erie $X_t$.
\newline
la s\^erie temporelle peut \^etre mod\'elis\'e par un processus {\em stationnaire} et {\em non stationnaire}.
% page 6 - 12 a comprendre et copier 
% master 1: cours de series temporelles notes de JM bardet
% introductionAuxSeriesTemporelles : page 42 : ts non stationnaire
% Gouriéroux C. et Montfort A. Cours de séries temporelles , Economica, Paris, 1983.
\subsubsection{Processus stationnaire}

\subsubsection{Processus non stationnaire}


% objectifs sur nos donnees
Conna\^itre ces param\^etres, en particulier la p\'eriode $T$, nous permet de selectionner des \'echantillon repr\'esentatifs des s\'eries dans le but de les comparer entre eux.
\section{Application des m\'ethodes sur les s\'eries temporelles de Champlan}
% quel est la methode choisi et quel est la periode trouvee. Puis que faire du bruit apres detendanciation et desaisonnalisation.

\end{document}