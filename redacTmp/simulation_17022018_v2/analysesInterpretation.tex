Les valeurs de corr\'elations sont d\'efinies par les lois uniformes (erreurs vrai positives et vrai n\'egatives) et exponentielles (erreurs fausses positives et fausses n\'egatives) comme illustr\'e par la figure \ref{distributionErreursCorrelations} pour une probabilit\'e d'ajout des erreurs $p\_correl = 0.5$. 
Cette valeur $p\_correl$  signifie qu'il y a autant d'erreurs fausses positives que d'erreurs fausses n\'egatives dans la matrice de corr\'elation du graphe $G_{k}$. 
Ces lois de distribution respectent l'hypoth\`ese de corr\'elations des arcs. Cette hypoth\`ese affirme que deux arcs correl\'es ont leur valeur de correction qui tende vers $1$ alors qu'une valeur proche de $0$ d\'esigne des arcs non correl\'es. 
\newline
% interpretation d'une distribution  DH et DL pour une methode
% presentation des differentes methodes et comparaison
% comparaison entre differentes p
% impact de la fonction de cout
Nous d\'ecrirons d'abord les distributions des distances line et de Hamming moyenn\'ees ($moy\_DL$ ou $moy\_DH$) pour une m\'ethode de correction (al\'eatoire). Ensuite nous comparons les cinq m\'ethodes de correction en nous basant sur les distances de Hamming moyenn\'ees. 
Enfin nous expliquons le choix de la m\'ethode de {\em permutation al\'eatoire} et montrons que les algorithmes (couverture et correction) proposent de meilleurs r\'esultats lorsque la matrice de corr\'elation poss\`ede plus de corr\'elations {\em faux n\'egatives} que de corr\'elations {\em faux positives} et aussi peu d'erreurs de corr\'elations ($k < 6$). 
\newline
Nous pr\'esentons \'egalement l'impact de la fonction de co\^ut dans les distributions  de distances de Hamming et 
la relation existante entre la distance line et la distance de Hamming.