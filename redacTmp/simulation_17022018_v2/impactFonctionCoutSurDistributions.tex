\label{fonctionDeCout}

\begin{figure}[htb!] 
\centering
% a changer par des chemins relatifs
\includegraphics[scale=0.25]{simulation_comparaisonDifferentesFonctDeCout_by_aleatoire_p_correl_05.jpeg}
\caption{ Comparaison des diff\'erentes fonctions de co\^ut avec l'ajout de $k \in [1,9]$ de corr\'elations fausses positives et fausses n\'egatives pour une probabilit\'e $p = 0.5$ avec la m\'ethode de permutation al\'eatoire }
\label{compareDifferentesFonctionDeCout_p05} 
\end{figure}

Nous d\'efinissons quatre fonctions de co\^ut: unitaire (n=0), normal (n=1), quadratique (n=2), puissance 4 (n = 4) selon l'expression suivante
\begin{equation}
	\begin{aligned}
	F_n = 
	\begin{cases}
		prob[(a_i,a_j)]^n   \\
		(1 - prob[(a_i,a_j)])^n \\
	\end{cases}
	\end{aligned}
\end{equation}
ou $prob[(a_i,a_j)]^n$, la corr\'elation entre les ar\^etes $a_i$ et $a_j$, correspondant au co\^ut d'ajout de corr\'elations fausses n\'egatives, 
$(1-prob[(a_i,a_j)])^n$ au co\^ut d'ajout de corr\'elations fausses positives.
\newline
\'Etant donn\'ee que nous avons utilis\'e des matrices binaires dans la g\'en\'eration de line-graphes, nous assignons des probabilit\'es pour chaque type de corr\'elation tel que:
\begin{itemize}
\item $prob[(a_i,a_j)] = [0, 0.5[ $ : corr\'elation vrai n\'egative i.e $0 \rightarrow 0$
\item $prob[(a_i,a_j)] = [0.5, 0.79[ $: corr\'elation fausse n\'egative i.e $1 \rightarrow 0$
\item $prob[(a_i,a_j)] = 0.8 $ : corr\'elation fausse positive i.e $0 \rightarrow 1$
\item $prob[(a_i,a_j)] = ]0.8, 1] $ : corr\'elation vrai positive i.e $1 \rightarrow 1$
\end{itemize}
La figure \ref{compareDifferentesFonctionDeCout_p05} affiche les courbes des diff\'erentes fonctions de co\^ut selon les $k$ corr\'elations modifi\'ees pour $p\_correl = 0.5$.
\newline
Les $4$ courbes sont superpos\'ees pour $k<5$ et au d\'el\`a de $k\ge5$, les courbes carr\'ee (n=2) et normal (n=1) ont les plus petites distances moyenn\'ees de Hamming.
Pour $k\ge9$, la courbe unitaire (n=0) \'evolue de mani\`ere exponentielle tandis que les courbes quadratique (n=2) et normal (n=1) ont un rapport de $2$ entre les distances de Hamming et le nombre $k$ de corr\'elations modifi\'ees.
On en d\'eduit que la pire fonction de co\^ut est la fonction unitaire (n=0 en jaune).
\newline
Cependant, nous ne pouvons pas choisir la meilleure fonction de co\^ut car les fonctions des courbes {\em normal} et {\em quadratique} \'evoluent identiquement.
 On peut conclure qu'utiliser les probabilit\'es de corr\'elations pour le calcul du co\^ut am\'eliore significativement les distances moyenn\'ees de Hamming.
 \newline
 On choisit pour la suite la fonction de co\^ut normal $F_1$ pour le co\^ut de modification de chaque ar\^ete. 

\paragraph{Cas particulier: fonction de co\^ut en cloche}

%petit rappel sur comment fonctionne la correction avec les autres fonctions de cout
Souvenons nous que corriger la matrice binaire de corr\'elation consiste \`a modifier les cases de cette matrice par leurs valeurs contraires.
Le cas id\'eal serait la modification des corr\'elations fausses positives et fausses n\'egatives pendant la phase de correction.
Cependant, pendant cette phase, la correction est effectu\'ee autant sur les corr\'elations fausses positives et fausses n\'egatives que sur celles vrai positives et vrai n\'egatives.
L'algorithme de correction ne fait aucune distintion entre les corr\'elations fausses et les bonnes corr\'elations.
Afin de prioriser la modification des corr\'elations fausses, nous definisons une nouvelle fonction de co\^ut appel\'ee {\em cloche}.
\newline
% definir la fonction de cloche et sa particularite
La fonction {\em cloche} se d\'efinit ainsi:
\begin{equation}
	F_c = | 4\cdot((prob-s) - (s-0.5))^2 |^{1.5}  
\end{equation}
La fonction de co\^ut cloche est une fonction polynomiale de degr\'e $2$ qui applique des poids aux ar\^etes de sorte que les ar\^etes, dont les corr\'elations tendent vers le seuil, sont moins p\'enalis\'ees et les ar\^etes, dont les corr\'elations sont \'eloign\'ees du seuil, ne sont quasi jamais utilis\'ees pendant la phase de correction. 
Les poids minimaux sont appliqu\'es pour les corr\'elations fausses tandis que les poids maximaux (= $1$) sont appliqu\'es aux bonnes corr\'elations
%La fonction de co\^ut cloche est une fonction polynomiale de degr\'e $2$ qui applique les co\^uts les plus \'elev\'es aux corr\'elations exactes (c'est-\`a-dire $1$) et les co\^uts les plus faibles aux corr\'elations fausses.
Cette fonction depend de la valeur de corr\'elation $prob$ entre deux ar\^etes et du seuil $s$ \`a partir duquel on introduit des corr\'elations erron\'ees.
\newline
% comparaison entre les fonctions de cout unitaire, lineaire et en cloche

Les algorithmes (couverture et correction) sont ex\'ecut\'ees avec la fonction cloche sur les graphes g\'en\'er\'es pr\'ec\'edemment. 
Les diff\'erentes distances line/Hamming obtenues, compar\'ees avec les fonctions de co\^uts unitaires $F_0$ et normales $F_1$, sont r\'esum\'ees dans la figure \ref{comparaison_fct_cloche_unitaire_normal_p05}.
En effet, pour $10$ et $20$ erreurs de corr\'elation, l'algorithme de correction modifie  de $11$ \`a $22$ ar\^etes pour la fonction normale $F_1$ et aussi de $38$ \`a $65$ pour la fonction en cloche respectivement. 
% expliquer
Nous en deduisons que le co\^ut de la fonction $F_1$ est  inf\'erieur \`a celui de la fonction en cloche. 
La figure \ref{comparaison_fct_cloche_unitaire_normal_p05}  est une bonne illustration par la position de la courbe rouge ($F_1$) en dessous de celle en jaune ($cloche$).
\newline
Ce r\'esultat provient de l'utilisation des ar\^etes fausses n\'egatives parce que le co\^ut de ces ar\^etes est faible par rapport au co\^ut des ar\^etes de corr\'elations fausses positives.
En effet les corr\'elations fausses n\'egatives et fausses positives appartiennent aux intervalles $[0,s[$ et $]s,1]$ respectivement. 
Alors que nous avons une d\'ecroissance faible de $0$ \`a $s$ et une croissance forte de $s$ \`a $1$ dans la courbe de $F_c$. 
Ce qui explique le co\^ut faible des fausses n\'egatives et celui \'elev\'e des fausses positives. 
Par ailleurs, la courbe en bleu (fonction unitaire $F_0$) est au dessus des deux autres courbes ($F_1$ et cloche) indiquant que son co\^ut est le plus \'elev\'e des trois fonctions.
\newline
%% ---- REPROGRAMMER
\begin{figure}[htb!] 
\centering
% a changer par des chemins relatifs
\includegraphics[scale=0.25]{simulation_comparaisonDifferentesFonctDeCout_by_aleatoire_p_05.jpeg}
\caption{ Comparaison des fonctions de co\^ut unitaire, normal et en cloche avec l'ajout de $k \in [1,9]$ de corr\'elations fausses positives et fausses n\'egatives pour une probabilit\'e $p = 0.5$ avec la m\'ethode de permutation al\'eatoire }
\label{comparaison_fct_cloche_unitaire_normal_p05} 
\end{figure}
%% ---- REPROGRAMMER
%Conclusion
On en conclut, par transitivit\'e, que la fonction $F_1$ fournit les co\^uts minimum pendant la correction.

% section de definition de parametres pour avoir le line graphe de distance de Hamming minimale.

