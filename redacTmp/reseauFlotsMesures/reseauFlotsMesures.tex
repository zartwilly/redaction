\documentclass[onecolumn, 12pt]{article}

\usepackage[latin1]{inputenc}   
\usepackage{amsmath}
\usepackage{algorithm}
\usepackage{algorithmic} 
%\usepackage[T1]{fontenc}

%\usepackage[francais]{babel}     
\usepackage{layout}    
\usepackage[top=2cm, bottom=2cm, left=2cm, right=2cm]{geometry} 
\usepackage{setspace}
\usepackage{soul}
\usepackage{color} 
\usepackage{verbatim}
\usepackage{moreverb}
\usepackage{listings}
\usepackage{url}
\usepackage{graphicx}
\usepackage[outdir=/home/willy/Documents/latexDoc/redactionThese/fusion_fichiers/images_fusionChapitres/]{epstopdf}
\usepackage{caption}
\usepackage{setspace}
\usepackage{amssymb} % used for not exists symbol ==> \nexists
\usepackage{amsthm}
 
\title{chapitre3: R\'eseau de flots et mesures physiques}
\author{Wilfried Ehounou}
\date{\oldstylenums{\today}} 

\newtheorem{definition}{D\'efinition}
\newtheorem{property}{Propri\'et\'e}
\newtheorem{theorem}{Theorem}
\newtheorem{claim}[theorem]{Claim}
\newtheorem{proposition}[theorem]{Proposition}
\newtheorem{lemma}[theorem]{Lemma}
\newtheorem{corollary}[theorem]{Corollary}
\newtheorem{conjecture}[theorem]{Conjecture}
\newtheorem{observation}{Observation}
\newtheorem{example}{Exemple}
\newtheorem{remark}{Remark}

%---insert paragraph (use 4) and subparagraph (use 5) to table of contents
\setcounter{tocdepth}{4} 
\setcounter{secnumdepth}{4}

%---- path figures ----
\graphicspath{{/home/willy/Documents/courbePython/courbeDegreCoutMinAleatoire_11_09_2017/}
{/home/willy/Documents/courbePython/courbeDegreCoutMinAleatoire_11_10_2017/}
{/home/willy/Documents/courbePython/courbeDegreCoutMinAleatoire_11_09_2017/comparaison_MethodesCorrection_fctDeCout_permut_aleatoire_coutMin_degreMin/}
{/home/willy/Documents/latexDoc/redactionThese/fusion_fichiers/images_fusionChapitres/}}
 
\begin{document}
\maketitle
\tableofcontents

\section{R\'eseau \'electrique d'un datacenter : un graphe de flots.}
% infrastructure d'un datacenter
Un Datacenter ou centre de donn\'ees est un site physique regroupant des installations informatiques (serveurs, r\'eseau informatique, syst\`eme de sauvegarde) et une infrastructure \'energetique ad\'equate (un syst\`eme de distribution \'electrique, un commutateur \'electrique, des r\'eserves d'\'energie, des g\'en\'erateurs d\'edi\'es au sauvegarde de donn\'ees, un syst\`eme de ventilation et de refroidissement).
Pour  fonctionner correctement, le datacenter doit abriter un r\'eseau \'electrique qui alimente \`a la fois les infrastructutes informatique et \'energetique. 
\newline
% caracteristique du reseau electrique
Le r\'eseau \'electrique se subdivise en trois parties
\paragraph{Les sources}: 
Ces \'equipements sont les points d'entr\'ee de l'\'electricit\'e dans le datacenter. 
Ils sont directement rattach\'es au gestionnaire de r\'eseau r\'egional ou national et sont de deux types : 
\begin{itemize}
\item Ceux qui alimentent le reste du r\'eseau en cas de dysfonctionnement du gestionnaire de r\'eseau. Ce sont les accumulateurs, les groupes \'electrog\`enes.
\item Ceux qui transforment la puissance re\c cue du gestionnaire pour les utiliser dans le datacenter. Ce sont les transformateurs basse tension communement appel\'es {\em Transformateur G\'en\'eral Basse Tension (TGBT)} 
\end{itemize}
\paragraph{Les tableaux}:
Aussi appel\'e tableau de r\'epartition, ce sont des \'equipements passifs dont leur fonction est celle de communateur. 
Ils repr\'esentent l'organe centrale de l'installation dans la mesure o\`u ils regroupent tous les circuits \'electriques et syst\`emes de protection vers les baies de serveurs. Chaque baie poss\`ede un disjoncteur sur ce tableau afin d'interrompre l'alimentation en cas de danger. 
\paragraph{Les baies (ou racks) de serveurs}:
Les baies distribuent la consommation de chaque serveur qui lui est rattach\'e. La baie a un r\^ole de multiprise pour tous les serveurs appartenant \`a cette derni\`ere. 
Dans le syst\`eme de supervision \'electrique, les baies sont les consommateurs de l'\'electricit\'e.
\newline
% reseau de flots
L'\'electricit\'e, achemin\'ee par le gestionnaire de r\'eseau, arrive aux transformateurs basse tensions qui g\'en\'eralement fonctionnent en mode triphas\'e.
Chaque phase repartit cette \'energie aux divers tableaux. 
Chaque tableau peut \^etre doublement rattach\'e \`a deux phases pour \'eviter les micro coupures d'\'electrict\'e. 
Les tableaux sont rattach\'es aux phases et aux baies par des c\^ables \'electriques. 
Les c\^ables sont unidirectionnels et aucun \'equipement ne s'alimente soit-m\^eme.
L'\'electricit\'e suit un sens : des sources vers les baies et aucune baie n'alimente une autre baie et un tableau. \newline
Notre r\'eseau \'electrique se  mod\'elise avec un r\'eseau de flots dont le graphe est un graphe acyclique {\em Directed Acyclic Graph (DAG)} puisque chaque \'equipement est repr\'esent\'e par un sommet, les c\^ables unidirectionnels r\'epresent\'es par des arcs et qu'aucun \'equipement ne s'alimente soi-m\^eme (absence de cycle dans le r\'eseau). 

\subsection{Topologie du r\'eseau \'electrique}
%  le graphe est un DAG
Le r\'eseau \'electrique est repr\'esent\'e par un graphe acyclique dont chaque arc a un flot et il est not\'e $G = (V, E, CAP)$. \newline
L'ensemble des sommets $V$ est compos\'e des \'equipements sources $V_S$, interm\'ediaires ou passifs $V_I$ et  charges ou serveurs $V_C$. L'ensemble des arcs $E$ se compose des c\^ables \'electriques reliant tous ses composants \'electriques. 
La r\'esistance de chaque c\^able est consider\'e constance ($r_i = cte $).
\newline
L'ensemble des sommets $V$ de cardinalit\'e $n$ est une union disjointe de $V_C$, $V_I$ et $V_S$ dont 
\begin{itemize}
	\item $V_S$ : sommets de degr\'e entrant null $d^- = 0$.
	\item $V_I$ : sommets de degr\'es entrant et sortant non nuls $d^- \ne 0, d^+ \ne 0$.  
	\item $V_C$ : sommets de degr\'e sortant null  $d^+ = 0$.
\end{itemize}
$$ V = V_S \cup V_I \cup V_C \hspace{0.4 em} et \hspace{0.4 em} V_S \cap V_I \cap V_C = \emptyset $$
L'ensemble des arcs $E$ de cardinalit\'e $m$ se d\'efinit ainsi :
\begin{equation}
 E = \{a_k = (v_i, v_j), \forall k \le m \} \hspace{0.4 em } avec \hspace{0.4 em} v_i \in V_i \subset V, v_j \in V_j =  V - V_i \hspace{0.4 em} et \hspace{0.4 em}  V_j \ne V_S 
\end{equation}
avec chaque arc ayant une r\'esistance constante. 
\newline
Chaque arc a une valeur maximale de flots pouvant le traverser. Cette valeur est la capacit\'e de l'arc.
L'ensemble des capacit\'es de cardinalit\'e $n$ (identique avec $E$) 

\subsubsection{Description du r\'eseau de flots}
Le r\'eseau \'electrique est constitu\'e d'un graphe acyclique et d'un ensemble de capacit\'es de cardinalit\'e $n$ (identique avec $E$). La capacit\'e est la valeur maximale du flot pouvant circuler sur un arc. Il est alors un r\'eseau de flots  $G = ( G', CAP)$

\subsection{Grandeurs physiques}
%les câbles sont des systèmes à constantes
%réparties, c'est à dire que ces grandeurs physiques sont réparties sur
%toute la longueur de la ligne.
\subsection{mesures physiques : TS}

\subsection{Contraintes ou regles : loi de kirrchoff}

\end{document}