\begin{definition}
Soient $G$ et $G'$ deux graphes ayant le m\^eme ensemble  de sommets.
%La distance en $G$ et $G'$ est la distance de Hamming not\'ee $DH(G,G')$ entre leurs deux matrices d'adjacente, c'est-\`a-dire le nombre d'\'el\'ements ayant une valeur diff\'erente dans chacune des deux matrices.
La distance de Hamming entre $G$ et $G'$ not\'ee $DH(G,G')$ est le nombre d'\'el\'ements ayant une valeur diff\'erente dans chacune des deux matrices d'adjacence.
\end{definition}
Une distance de Hamming \'egale \`a $k$ $(k \in \mathbb{N})$ signifie qu'il existe $k$ ar\^etes diff\'erentes entre  les graphes $G$ et $G'$.

\begin{definition}
On appelle distance-line de $G$, not\'ee $DL(G)$, la plus petite distance de Hamming entre $G$ et $G'$, $G'$ \'etant le line graphe d'un DAG ayant le m\^eme ensemble de sommets que $G$.
\end{definition}

Nous consid\'erons le probl\`eme suivant. \newline
{\bf Probl\`eme} Proxi-Line \newline
{\bf Donn\'ees} : Un graphe $G=(V,E)$, un entier $k$. \newline
{\bf Question} : $DL(G) \le k$ ? \newline

\begin{conjecture}
Proxi-Line est NP-complet.
\end{conjecture}

La preuve de la conjecture est en annexe 1.