% ----
% decouverte de topologie du reseau energetique
% ----
Soient $\cal C$, l'ensemble des cliques de la line graphe $G_M$ et $G$ le graphe racine de $G_M$. \newline
Chaque clique $C_k \in {\cal C}$ correspond \`a un sommet du graphe racine $G$. \newline 
Soit $u = \{``ENTRANT'', ``SORTANT'', ``NONE''\}$, l'ensemble des marquages de chaque ar\^ete tel que les \'etiquettes $``ENTRANT''$ et $``SORTANT''$ sont associ\'ees respectivement \`a l'ar\^ete $a_i$ entrante et sortante du sommet $C_k$.
\newline
%-- definition de S_1 e S_2
Soient $S_1$ et $S_2$ l'ensemble des arcs entrants et sortantes du sommet $C_k$.
\begin{equation}
S_\alpha = \{ S_{\alpha,i}, \hspace{0.2 em} \forall i \le 2^{card(C_k)+1}  \}
\end{equation}
avec $i$, le nombre de sous-ensemble de $C_k$ et $\alpha = \{1,2\}$. Le sous-ensemble $S_{\alpha,i}$ peut \^etre  l'ensemble vide $\emptyset$.
 
 %--  ORACLE
 \begin{definition}
 Soit la clique $C_k$ de la couverture {\cal C}.
 Un couple $(S_1,S_2)$ de sous-ensembles de $C_k$ est {\bf valide} si 
 \begin{itemize}
 \item $S_1 \ne S_2$
 \item $S_1 \cup S_2 = C_k$
 \item $S_1 \cap S_2 = \emptyset$
 \end{itemize}
 \end{definition}
 
 \begin{definition}
 Soit une fonction bool\'eenne $ORACLE$ d\'efinit de $S_1 \times S_2 \rightarrow \{0,1\}$, avec le couple valide $(S_1,S_2)$.
 La fonction $ORACLE$ renvoie $1$ si et seulement si la loi de conservation autour du sommet $C_k \in G$ est respect\'ee c'est-\`a-dire : 
 \begin{equation}
 \forall a_i \in S_1, \forall a_j \in S_2, \sum_{a_i \in S_1} gp(a_i) - \sum_{a_j \in S_2} gp(a_j) \le \epsilon
 \end{equation}
 avec $\epsilon$ les pertes minimales par effets joules, $gp(a_i)$ le flot dans l'arc $a_i \in E(G_M)$.
 \end{definition}
 
 \begin{property}
 Si $ORACLE(S_1, S_2) = 1$ alors l'ensemble  $S_1$ est l'ensemble des arcs entrants et  $S_2$ l'ensemble des arcs sortants du sommet $C_k \in G$.
 \end{property} 
Les arcs de $S_1$ et $S_2$ ne concourent pas \`a un sommet $C_k$ lorsque  $ORACLE(S_1, S_2) = 0$.

\begin{theorem}
\label{arcsentrantsSortants}
%Un arc $a_i$ est soit {\em entrant}, soit {\em sortant} mais jamais les deux.
Si un sommet $a_i \in G_M$ appartient \`a deux cliques $C_{k_1}, C_{k_2} \in {\cal C}$ alors 
l'arc  $a_i \in E(G_M)$ est soit entrant de $C_{k_1}$  et sortant de $C_{k_2}$ ou soit entrant de $C_{k_2}$  et sortant de $C_{k_1}$.
\end{theorem}

\begin{proof}
D'apr\`es le theoreme \ref{caracteristiquesLinegraphes}, le sommet $a_i$ appartient \`a deux cliques $C_{k_1}$ et $C_{k_2}$ au maximum. 
Cela implique qu'il existe une ar\^ete entre les sommets $C_{k_1}$ et $C_{k_2}$ dans $G$.
Soient $S_1, S_2 \subset C_{k_1}$ et $S_1, S_2 \subset C_{k_2}$ telles que $ORACLE(S_1, S_2) = 1$ et $ORACLE(S_3, S_4) = 1$.
Comme  $S_1 \cap S_2 = \emptyset$, $S_3 \cap S_4 = \emptyset$ et aussi  $C_{k_1} \cap C_{k_2} = \{a_i\}$, on a quatre choix possibles :
$a_i \in S_1 \cap S_3$, $a_i \in S_1 \cap S_4$, $a_i \in S_2 \cap S_3$, $a_i \in S_2 \cap S_4$.
\newline
Si $a_i$ est entrant et sortant \`a $C_{k_1}$ et $C_{k_2}$ alors $a_i \in S_1 \cap S_3$ ou  $a_i \in S_2 \cap S_4$. 
On en d\'eduit qu'il existe une boucle sur le sommet $C_{k_1}$ et $C_{k_2}$. 
Cela est impossible parce que $G$ est un $DAG$. 
\newline
Alors $a_i$ est entrant \`a $C_{k_1}$ $=>$ $a_i \in S_2 \cap S_3$ ou  $a_i$ sortant \`a $C_{k_2}$  $a_i \in S_1 \cap S_4$. 
\end{proof}

Soit $A$ l'ensemble des cliques $C_j$ tel que $card(C_j) = min\{ card(C_k), C_k \in {\cal C}\}$.
\newline
Soit la fonction $f$ d\'efinie sur $ V(G_M) \times A \rightarrow \{0,1\}$ retournant $1$ si le sommet $a_i^u$ ne poss\`ede aucune \'etiquette c'est-\`a-dire labellis\'e \`a $None$.
$$ f(a_i^u, C_j) = \begin{cases} 1 \hspace{0.2 em} si \hspace{0.2 em} M(a_i^u, C_j) \ne None \\ 0 \hspace{0.2 em} sinon \end{cases}$$
La fonction $MA$ calcule le nombre de sommets labellis\'e dans une clique $C_k$.
$$MA(C_k) =  \sum_{a_i^u \in C_k} f(a_i^u,C_k) $$
Soient la clique $C_m$ telle que  $MA(C_m) = max\{ MA(C_j), C_j \in A\}$ et $B$ l'ensemble des sommets de la clique $C_m$ tel que $M(C_m, a_j^v) == None, \forall a_j^v \in B$. 


%nommer l'algorithme
\begin{algorithm}[!ht]
\caption{Decouverte\_graphe\_racine}
\noindent DEBUT\\
\noindent 1. Initialisation des \'etiquettes des sommets $a_i$ de $G_M$, et $C_k \in {\cal C}$ \\
\noindent $M( C_k, a_i^u) = None$ \\
~2. \indent {\bf Tant que} il existe une  ar\^ete de $G_M$ ($E(G_C)$) \\
	\indent~~~~~~{\bf Faire}\\
~3.		\indent~~~~~~~~choisir $A = \{C_j , card(C_j) = min\{ card(C_k), C_k \in {\cal C} \}\}$ \\
~4.		\indent~~~~~~~~choisir $C_k$ tel que $MA(C_k) = max\{ MA(C_j), C_j \in A\} $ \\
~5.       	\indent~~~~~~~~{\bf Si} il existe $S_1$ et $S_2$ tel que $S_1 \cup S_2 = \emptyset $ et $S_1 \cup S_2 = C_k $ et $ORACLE(S_1, S_2, M )= 1 $ \\
~6.	       	\indent~~~~~~~~~~~~{\bf alors}\\
~7.	       	\indent~~~~~~~~~~~~ $S_1^k = sommets\_marques(S_1, M)(^1)$; \\
~8.	       	\indent~~~~~~~~~~~~ $S_2^k = sommets\_marques(S_2, M)$; \\
~9.		\indent ~~~~~~~~~~~~{\bf Pour tout} $a_i^K \in S_1 - S_1^k$ {\bf Faire} \\
~10.		\indent ~~~~~~~~~~~~~ $M(C_k, a_i^K) = ``ENTRANT''$ \\
~11.		\indent ~~~~~~~~~~ {\bf Pour tout} $a_i^K \in S_2 - S_2^k$ {\bf Faire} \\
~12.		\indent ~~~~~~~~~~~~~ $M(C_k, a_i^K) = ``SORTANT''$ \\
~13.		\indent~~~~~~~~~~ $E(G_{C}) =  E(G_>)  - E(G_M[C_k])$ \\
~14.		\indent~~~~~~~~~~ ${\cal C} =  {\cal C}  - C_k$ \\
~15.       	\indent~~~~~~~~{\bf FinSi} \\
~16. \indent {\bf FinTant que} \\
~17. \indent {\bf Pour tout} $a_i^u \in V(G_M)$ {\bf Faire} \\
~18. \indent ~~~ $C_1, C_2$ = $Couverture\_Cliques(a_i^u, M)(^2)$\\
~19. \indent ~~~ {\bf Si } $C_1 \neq \emptyset$ et $C_2 \neq \emptyset$ {\bf Alors} \\
~20. \indent ~~~~~~  {\bf Si}  $M(C_1, a_i^u) == ``SORTANT''$ {\bf Alors} \\
~21. \indent ~~~~~~~~~~~~ $Mat(C_1)$ += $C_2$ $(^3)$ \\  
~22. \indent ~~~~~~  {\bf Fin Si} \\
~23. \indent ~~~~~~  {\bf Si}  $M(C_2, a_i^u) == ``SORTANT''$ {\bf Alors} \\
~24. \indent ~~~~~~~~~~~~ $Mat(C_2)$ += $C_1$ $(^3)$ \\  
~25. \indent ~~~~~~  {\bf Fin Si} \\
~26. \indent ~~~ {\bf Fin si} \\
~27. \indent ~~~ {\bf Si } $C_1 \neq \emptyset$ et $C_2 == \emptyset$ {\bf Alors} \\
~28. \indent ~~~~~~  {\bf Si}  $M(C_1, a_i^u) == ``ENTRANT''$ {\bf Alors} \\
~29. \indent ~~~~~~~~~~~~ $Mat(EXT\_a_i^u)$ += $C1$ $(^3)$\\  
~30. \indent ~~~~~~  {\bf Fin Si} \\
~31. \indent ~~~~~~  {\bf Si}  $M(C_1, a_i^u) == ``SORTANT''$ {\bf Alors} \\
~32. \indent ~~~~~~~~~~~~ $Mat(C_1)$ += $EXT\_a_i^u$ $(^3)$ \\  
~33. \indent ~~~~~~  {\bf Fin Si} \\
~34. \indent ~~~ {\bf Fin si} \\
~35. \indent {\bf Fin Pour} \\
~36. \noindent {\bf Return} Mat\\
\noindent FIN\\
\end{algorithm}

$^1$ : Cette fonction retourne les sommets de l'ensemble $S_1$ ou $S_2$ labellis\'es \`a $``ENTREE''$ ou $``SORTANT''$. Initialement, Ces sommets sont \'etiquett\'es \`a $``None''$.

$^2$ : Cette fonction renvoie les cliques couvrantes une ar\^ete $a_i^u$ avec $C_1 \ne \emptyset$. Si $a_i^u$ est couvert par une clique alors $C_2 = \emptyset$.

$^3$ : $EXT\_a_i^u$ est un sommet du graphe racine $G$.
\newline

L'algorithme {\em Decouverte\_graphe\_racine} d\'ebute par l'initialisation des sommets de $G_M$ \`a $''None''$.
Tant qu'il existe une ar\^ete dans notre line-graphe $G_C$, on choisit la plus petite clique $C_k$ dont le maximum de sommets de cette clique est labellis\'e soit par $''ENTRANT''$ ou soit par $''SORTANT''$ (lignes $3-4$).
On partitionne la clique $C_k$  en deux sous-ensembles valides $S_1$ et $S_2$ tels que $S_1$ et $S_2$ correspondent, respectivement, \`a l'ensemble des arcs entrants et sortants du graphe racine $G$.
Les sommets labellis\'es de $S_1$ not\'es $S_1^k$ portent l'\'etiquette $''ENTRANT''$ tandis que ceux labellis\'es en $''SORTANT''$ sont not\'es $S_2^k$ (lignes $6-12$). 
Ensuite nous mettons \`a jour l'ensemble des ar\^etes et la line-couverture de $G_C$ (lignes $13-14$). 
\newline
Une fois termin\'e l'orientation des ar\^etes de $G$ qui  sont les sommets labellis\'es du line-graphe $G_M$ (lignes $2-16$), nous construisons la liste d'adjacence de chaque sommet de $G$ (lignes $17-36$).
Nous recherchons la couverture d'un sommet $a_i^u$ de $G_M$.
\newline 
Si le sommet  $a_i^u$ est couvert par une seule clique alors $C_2 = \emptyset$. Dans ce cas, cela signifie qu'il existe un arc entre le sommet $C_1$ et le sommet $EXT\_a_i^u$ que nous avons cr\'ee. Cet arc a pour extr\'emit\'e initiale $C_1$ si $a_i^u$ est labellis\'e par $''SORTANT''$ sinon pour extr\'emit\'e initiale $EXT\_a_i^u$ si $a_i^u$ est labellis\'e par $''ENTRANT''$ (lignes $27-34$).
\newline
Dans le cas o\`u le sommet $a_i^u$ est couvert par deux cliques non vides $C_1, C_2$, nous ajoutons un arc entre ces deux cliques.
D'apr\'es le th\'eor\`eme  \ref{arcsentrantsSortants} stipulant qu'un sommet de $G_C$ est $''ENTRANT''$ de $C_1$ et $''SORTANT''$ de $C_2$ et vice-versa, nous d\'efinissons  $C_1$ comme extr\'emit\'e initiale de cet arc si $a_i^u$ est labellis\'e par $SORTANT$ dans cette clique $C_1$ sinon $C_2$ si $a_i^u$ est labellis\'e par $SORTANT$ dans cette clique $C_2$ (lignes $19-26$).

\subsection{Complexit\'e de l'algorithme {\em Decouverte\_graphe\_racine} }

La fonction $Couverture\_Cliques$ a une complexit\'e constante $O(1)$ alors que la complexit\'e de la  fonction $sommets\_marques$ depend du nombre de sommets dans $S_1$. Dans le pire des cas, sa complexit\'e est $O(n)$ avec $n = E(G)$ le nombre  d'arcs de $G$. Donc les lignes $17-35$ s'ex\'ecutent en $O(n)$ dans le pire des cas. 
\newline
Consid\'erons une line-couverture ${\cal C}$  de taille $K$,  une clique $C_i \in {\cal C}$ de taille $p_i$ et $k_i$ le nombre de sommets marqu\'es dans $C_i$. 
\newline
Le nombre de couples valides $(S_1,S_2)$ pour la clique $C_i$ avec $k_i$ sommets marqu\'es est $2^{p_i -k_i}$. 
Au traitement de la premi\`ere clique de ${\cal C}$ de taille minimale $C_1$, il existe $0$ sommets marqu\'es. Le co\^ut $R_{cv_1}$ de couples valides est $R_{cv_1} = 2^{p_1}$.
Au traitement de la seconde clique $C_2$, il existe $k_2 \ge 0$ et son co\^ut est  $R_{cv_2} = 2^{p_2 - k_2}$.
Il existe une clique $C_\alpha$ \`a partir de laquelle $k_\alpha > 0, \alpha \le K$.
On remarque que $p_\alpha - k_\alpha$ est d\'ecroissant car $k_\alpha$ augmente apr\`es chaque traitement de cliques. 
Cela signifie qu'\`a la selection de la derni\`ere clique $C_k$, son co\^ut  est de $R_{cv_K} = 1$. 
Le co\^ut est decroissant \`a chaque traitement c'est-\`a-dire 
$$2^{p_{\alpha} - k_{\alpha}} \ge 2^{p_{\alpha+1} - k_{\alpha+1}} \ge \cdots \ge 2^{p_K - k_K}$$ 
Le co\^ut des couples valides de ${\cal C}$ est 
$$R_{cv} = O(2^\gamma), \gamma = max\{card(C_i), \forall C_i  \in \{C_1, \cdots, C_\alpha\} \}$$
car il depend de la clique $C_i \in \{C_1, \cdots, C_\alpha\} \subset {\cal C}$.
\newline
Le co\^ut des lignes $3-4$ depend de $p_i$ et $K$ ($O(p_i) + O(K)$) et celui des lignes $7-14$ est de $O(p_i^2)$.
\newline
Soit le co\^ut $R$ de la boucle  {\em Tant que}. Il est de 
$$ R = O(p_i) + O(K) + O(p_i^2) + O(2^\gamma) \simeq O(2^\gamma)$$

La complexit\'e de {\em Decouverte\_graphe\_racine} est {\em pseudo-exponentielle} car l'exposant $\gamma$ est la taille d'une clique de taille interm\'ediaire.

%Consid\'erons les instructions dans le boucle {\em Tant que}, $K$ le cardinal de ${\cal C}$ et $p$ le cardinal de la clique de taille minimale de ${\cal C}$ c'est-\`a-dire $card(A) = p$. 
%\newline
%$R_i$ est le co\^ut de la boucle {\em Tant que} au traitement de la $i^eme$ clique de $G_C$ et $C\_oracle$ le co\^ut de la fonction $ORACLE$.
%S\'electionnons le premier clique de $G_C$. Son co\^ut est :
%$$R_1 = p \times C_oracle(S_{1p}, S_{2p}) $$.
%Le co\^ut  du deuxi\`eme sommet est :
%$$R_2 = p \times C_oracle(S_{1p}, S_{2p})  + (p-1) \times C_oracle(S_{1p-1}, S_{2p-1})  $$.
%Le co\^ut des $n$ sommets de $G_C$ est :
%$$R_K = \sum_{k = 0}^{K-1} (p-k) \times C_oracle(S_{1p-k}, S_{2p-k})$$
%
%La particularit\'e de l'ORACLE est que pour $k \ge p$, tous les sommets de la clique $C_k$ sont labellis\'es entrainant qu'on ne genere aucuns sous-ensembles de $C_k$. Cela implique que  $C_oracle(S_{1p-k}, S_{2p-k}) = 1$. \newline
%Cela revient \`a dire que le co\^ut $R_K$ est d\'ecroissant en fonction de $K$. \newline
%G\'en\'erer tous  sous-ensembles de $C_k$ de taille $p$ est une combinaison de 
%$card(P(E)) = 2^{p}$ et toutes les paires de $P(E)$ vaut au pire des cas $\frac{2^p * (2^p - 1)}{2}$. 
%Alors  le  cout  $C_oracle(S_{1p}, S_{2p}$ est $O(2^p)$ ===> FAUX
%%%%% A discuter avec SERGES ---> trouver le nombre de tuples provenant des sous ensembles  de C_k de taille $p$
%Donc le co\^ut de $R_K$ est  
%$$R_K = p \times O(2^p) + (p-1) \times O(2^{(p-1)}) + \cdots + 1 \times O(2^0)$$
%$$ R_K $$
%PAS FINI


