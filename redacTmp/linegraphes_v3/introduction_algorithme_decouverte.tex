Diff\'erents travaux ont \'et\'e r\'ealis\'es sur la d\'ecouverte de line-graphes.
Parmi lesquelles, nous citons  l'algorithme de ROUSSOPOULOS \cite{ROUSSOPOULOS1973108} qui utilise une propri\'et\'e des line-graphes provenant des travaux de KRAUSZ \cite{krausz1943demonstration}. 
Il affirme que le graphe $G$ est un line-graphe si ces ar\^etes  peuvent \^etre partition\'ees en cliques  de tel sorte qu'aucun sommet ne soit couvert par plus de deux cliques. 
Cette algorithme non seulement d\'etecte si $G$ est un line-graphe mais fournit son graphe racine en temps lin\'eaire $O(max(\{m,n\}))$, avec $m$ et $n$ le nombre de sommets et d'ar\^etes respectivement.
Un autre algorithme, propos\'e par Klauss Simon et Daniele Degiorgi \cite{decompositionEnCliques}, est une simplication du probl\`eme de reconnaissance de line-graphes. Bas\'e sur la preuve de ORE \cite{ORE} du th\'eor\`eme de Whitney \cite{whitney1932congruent}, il stipule que deux graphes connexes avec plus de quatre sommets sont isomorphes en ar\^etes s'il existe exactement un isomorphisme en sommets qui induit \'egalement un isomorphisme en ar\^etes. Il v\'erifie en un temps lin\'eaire une reconnaissance incrementale  sommet par sommet c'est-\`a-dire une modification locale (i.e l'ajout et la suppression) pr\'eserve les caract\'eristiques du line-graphe.
L'algorithme de Lehot \cite{decompositionEnCliquesParArcs} se sert aussi de la reconnaissance locale et a une complexit\'e en $O(n) + E$ avec $n$ le nombre de sommets de $G$ et $E$ le nombre d'ar\^etes de $L(G)$.
Il utilise le th\'eor\`eme par VAN ROOIJ and WILF \cite{ROOIJetWILF1965interchange} \'enoncant qu'un graphe $G$ est un line-graphe si $G$ ne contient pas de sous graphe induit $K_{1,3}$ et si deux graphes triangles ``Odd'' ont une ar\^ete commune, le sous graphe induit par ces sommets est une clique $K_4$. Rappelons qu'une graphe triangle $\{a_1,a_2,a_3\} \subseteq V(L(G))$ est ``Odd'' s'il existe un sommet $e \in V(G)$ incident \`a  au moins un des sommets $\{a_1, a_2, a_3\}$. Ce triangle est ``Even'' dans le cas contraire. 
\newline
Tous les algorithmes existants se limitent \`a un r\'esultat n\'egatif lorsque le graphe $G$ n'est pas un line-graphe. 
Cependant, l'article de {\em Halld{\'o}rsson and al.} \cite{Halldorsson2013} a pour but de corriger un graphe pour en obtenir un line-graphe. En effet, il propose une m\'ethode de d\'ecouverte de la g\'en\'ealogie de population en se basant sur les haplotypes partag\'ees dans les genomes des individus.  Les haplotypes sont uniques et sont les sommets d'un graphe dit {\em Clark Consistency graph} \cite{halldorsson2011clark} et les ar\^etes de ce graphe sont form\'ees par  des individus partageant les m\^emes haplotypes. 
L'hypoth\`ese sous-jacente de cet algorithme propos\'e par {\em Halld{\'o}rsson and al.} est l'existence de sommets surperflus dans le Clark Consistency graphe et cela entraine que ce graphe n'est pas un line-graphe. 
Ce probl\`eme \'etant {\em NP-Complete}, la solution propos\'ee r\'ealise un algorithme de suppression de sommets et d'ar\^etes. 
L'algorithme de suppression de sommets est une 6-approximation alors que celui des ar\^etes est de complexit\'e $O(n*m)$ avec $m$ le nombre de sommets et $n$ le nombre d'ar\^etes.
Dans le cas o\`u des suppressions sont effectu\'ees, le line-graphe fourni est le plus proche possible du line-graphe de l'arbre g\'en\'ealogique.
La particularit\'e de la solution est l'absence d'ar\^etes ajout\'ees dans le line-graphe et cela implique que cette solution est inapplicable dans notre probl\`eme o\`u il existe des ar\^etes inconnues dans notre graphe de corr\'elation.
\newline
Nous nous basons sur l'algorithme de Lehot parce qu'il s'ex\'ecute en un temps lin\'eaire en \'effectuant un traitement sommet par sommet pour la reconnaissance de sous-graphes complets. Ce traitement permet de s\'electionner les cliques existantes et les sommets, n'appartenant \`a aucune clique, qui n\'ecessitent une modification de leur voisinage.