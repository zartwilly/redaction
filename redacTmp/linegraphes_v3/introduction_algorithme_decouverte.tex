Diff\'erents travaux ont \'et\'e realis\'es sur la d\'ecouverte de line-graphes.
Parmi lesquelles, nous citons  l'algorithme de ROUSSOPOULOS \cite{ROUSSOPOULOS1973108} qui utilise une propri\'et\'e des line graphes provenant des travaux de KRAUSZ \cite{krausz1943demonstration}. 
Il affirme que le graphe $G$ est un line-graphe si ces ar\^etes  peuvent \^etre partition\'ees en cliques  de tel sorte qu'aucun sommet ne soit couvert par plus de deux cliques. 
Cette algorithme non seulement d\'etecte si $G$ est un line graphe mais fournit son graphe racine en temps lin\'eaire $O(max(\{m,n\}))$, avec $m$ et $n$ le nombre de sommets et d'ar\^etes respectivement.
Un autre algorithme, propos\'e par Klauss Simon et Daniele Degiorgi \cite{decompositionEnCliques}, est une simplication du probl\`eme de reconnaissance de line graphes. Bas\'e sur la preuve de ORE \cite{ORE} du th\'eor\`eme de Whitney \cite{whitney1932congruent}, il stipule que deux graphes connexes avec plus de quatre sommets sont isomorphes en ar\^etes s'il existe exactement un isomorphisme en sommets qui induit \'egalement un isomorphisme en ar\^etes. Il v\'erifie en un temps lin\'eaire une reconnaissance incrementale  sommet par sommet c'est-\`a-dire une modification locale (i.e l'ajout et la suppression) pr\'eserve les caract\'eristiques du line-graphe.
L'algorithme de Lehot \cite{decompositionEnCliquesParArcs} se sert aussi de la reconnaissance locale et a une complexit\'e en $O(n) + E$ avec $n$ le nombre de sommets de $G$ et $E$ le nombre d'ar\^etes de $L(G)$.
Il utilise le th\'eor\`eme par VAN ROOIJ and WILF \cite{ROOIJetWILF1965interchange} \'enoncant qu'un graphe $G$ est un line graphe si $G$ ne contient pas de sous graphe induit $K_{1,3}$ and si deux graphes triangles ``Odd'' ont une ar\^ete commune, le sous graphe induit par ces sommets est une clique $K_4$. Rappelons qu'une graphe triangle $\{a_1,a_2,a_3\} \subseteq V(L(G))$ est ``Odd'' s'il existe un sommet $e \in V(G)$ incident \`a  au moins un des sommets $\{a_1, a_2, a_3\}$. Ce triangle est ``Even'' dans le cas contraire. 
\newline
Tous les algorithmes existants se limitent \`a un r\'esultat n\'egatif lorsque le graphe $G$ n'est pas un line-graphe. N\'eammoins, nous nous baserons sur l'algorithme de Lehot parce qu'il s'ex\'ecute en un temps lin\'eaire en r\'ealisant un traitement sommet par sommet pour la reconnaissance de sous-graphes complets. Ce traitement nous permettra de s\'electionner les cliques existantes puis les sommets qui n\'ecessiteront une modification de leur voisinage.